\chapter{DBMS - Data Base Managament System}%
\label{cha:DBMS - Data Base Managament System}
\section{Sistema di gestione di basi di dati}%
\label{sec:Sistema di gestione di basi di dati}
I sistemi di gestione di basi di dati si pongono di risolvere i problemi legati
alla gestione e salvataggio di dati, quali:
\begin{itemize}
  \item \textbf{ridondanza ed inconsistenza dei dati}, ovvero la duplicazione
    dei dati su file multipli, che comporta anche un pericolo di inconsistenza;
  \item \textbf{difficoltà nell'accesso ai dati}, la mancanza di una descrizione
    di alto livello e centralizzata dei dati ne rende estremamente difficoltoso
    l'utilizzo al fine di rispondere a nuove esigenze applicative, a questo
    proposito i \emph{DBMS} forniscono linguaggi per facilitare l'accesso ai
    dati, integrazione con i linguaggi di programmazione e funzionalità
    reattive;
  \item \textbf{problemi nell'accesso concorrente ai dati}, operazioni multiple
    sullo stesso dato possono causare problemi di concorrenza, i \emph{DBMS}
    mettono a disposizione il concetto di \textit{transazione}, che garantisce
    la consistenza dei dati in presenza di transazioni concorrenti;
  \item \textbf{problemi di integrità dei dati}, i \emph{DBMS} forniscono la
    possibilità di stabilire dei vincoli al salvataggio ed alla modifica dei
    dati, preservandone l'integrità.
\end{itemize}

\subsection{Obiettivi e servizi di un DBMS}%
\label{sub:Obiettivi e servizi di un DBMS}
I DBMS implementano una serie di servizi per offrire l'utilizzo della base di
dati e lo sviluppo di applicazioni con cui si interfacciano, alcuni sono
invocabili dagli utenti, altri sono utilizzati per il funzionamento interno.

I DBMS adottano un'\textit{architettura client-server}, per cui le funzionalità
del sistema sono realizzate da due moduli distinti: il \textbf{client} che
gestisce l'interazione tra utente e DBMS ed il \textbf{server} che si occupa
della memorizzazione e gestione dei dati.

\subsubsection{Principali servizi offerti da un DBMS}%
\label{ssub:Principali servizi offerti da un DBMS}
\begin{itemize}
  \item \textbf{Descrizione dei dati:} per specificare i dati da memorizzare
    nella base di dati;
  \item \textbf{manipolazione dei dati, per:}
    \begin{itemize}
      \item accedere ai dati;
      \item inserire nuovi dati;
      \item modificare dati esistenti;
      \item cancellare dati esistenti;
    \end{itemize}
  \item \textbf{controllo di integrità:} per evitare di memorizzare dati non
    corretti;
  \item \textbf{strutture di memorizzazione:} per rappresentare in memoria
    secondaria i costrutti del modello dei dati;
  \item \textbf{ottimizzazione di interrogazioni:} per determinare la strategia
    più efficiente per accedere ai dati;
  \item \textbf{protezione dei dati:} per proteggere i dati da accessi non
    autorizzati;
  \item \textbf{ripristino della base di dati}, per evitare che errori e
    malfunzionamenti:
    \begin{itemize}
      \item determinino una base di dati inconsistente;
      \item provochino perdite di dati;
    \end{itemize}
  \item \textbf{controllo della concorrenza:} per evitare che accessi
    concorrenti alla base di dati provochino inconsistenze dei dati.
\end{itemize}
