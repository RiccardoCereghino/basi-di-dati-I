\chapter{DBMS - Data Base Managament System}%
\label{cha:DBMS - Data Base Managament System}
\section{Sistema di gestione di basi di dati}%
\label{sec:Sistema di gestione di basi di dati}
I sistemi di gestione di basi di dati si pongono di risolvere i problemi legati
alla gestione e salvataggio di dati, quali:
\begin{itemize}
  \item \textbf{ridondanza ed inconsistenza dei dati}, ovvero la duplicazione
    dei dati su file multipli, che comporta anche un pericolo di inconsistenza;
  \item \textbf{difficoltà nell'accesso ai dati}, la mancanza di una descrizione
    di alto livello e centralizzata dei dati ne rende estremamente difficoltoso
    l'utilizzo al fine di rispondere a nuove esigenze applicative, a questo
    proposito i \emph{DBMS} forniscono linguaggi per facilitare l'accesso ai
    dati, integrazione con i linguaggi di programmazione e funzionalità
    reattive;
  \item \textbf{problemi nell'accesso concorrente ai dati}, operazioni multiple
    sullo stesso dato possono causare problemi di concorrenza, i \emph{DMBS}
    mettono a disposizione il concetto di \textit{transazione}, che garantisce
    la consistenza dei dati in presenza di transazioni concorrenti;
  \item \textbf{problemi di integrità dei dati}, i \emph{DBMS} forniscono la
    possibilità di stabilire dei vincoli al salvataggio ed alla modifica dei
    dati, preservandone l'integrità.
\end{itemize}
