\section{Controllo dell'accesso}%
\label{sec:Controllo dell'accesso}
Il controllo dell'accesso regola le operazioni che è possibile compiere sui dati
e le risorse in una base di dati; lo scome è di limitare e controllare le
operazioni che gli utenti, già riconosciuto dal meccanismo di autenticazione,
effettuano, prevenendo azioni accidentali o deliberate che potrebbero
compromettere la correttezza e la sicurezza dei dati.

Nel controllo dell'accesso distinguiamo tre entità fondamentali: gli oggetti, i
soggetti ed i privilegi.

I soggetti possono essere ulteriormente classificati in: utenti, gruppi e ruoli.

La concessione o meno di un determinato accesso deve rispecchiare le
\emph{politiche di sicurezza} dell'organizzazione.

Le autorizzazioni possono essere rappresentate mediante una tupla $(s,o,p)$ dove
$s$ p il soggetto, $o$ l'oggetto e $p$ il privilegio.

Il controllo dell'accesso è effettuato mediante il meccanismo di
\emph{constrollo dell'accesso}, basato su un \emph{modello di controllo
dell'accesso}, (\emph{reference monitor}) il cui compito è di intercettare ogni
comando inviato al DBMS e stabilire, tramite l'analisi delle autorizzazioni, se
il soggetto richiedente puà essere autorizzato o meno a compiere l'azione
richiesta.
