\section{Algebra relazionale}%
\label{sec:Algebra relazionale}
L'algebra relazionale è costituita da $5$ operazioni per la manipolazione
delle relazioni:
\begin{itemize}
  \item proiezione;
  \item selezione;
  \item prodotto cartesiano;
  \item unione;
  \item differenza.
\end{itemize}

Ogni operazione ha come argomento una o più relazioni.
Esistono operazioni addizionali che possono essere espresse in termini delle
operazioni di base.

\subsection{Operazioni di base}%
\label{sub:Operazioni di base}
Sicchè facciamo riferimento alla notazione con nome, introduciamo l'operazione
di ridenominazione.

\paragraph{Ridenominazione}%
\label{par:Ridenominazione}
La ridenominazione di una relazione $R$ rispetto ad una lista di coppie di nomi
di attributi $(A_1,B_1),\dots,(A_m,B_m)$, tale che $A_i\in U_R$ è un nome di
attributo di $R$ indicata con $\rho_{A_1,\dots,A_m\leftarrow B_1,\dots B_m}(R)$,
ridenomina l'attributo di nome $A_i$ con il nome $B_i$, $i=1,\dots,m$.

La ridenominazione è corretta se il nuovo schema di relazione per $R$ ha
attributi con nomi tutti distinti.
La relazione ottenuta ha lo stesso grado della relazione $R$ ed ha lo stesso
contenuto.
Gli attributi della relazione risultato sono
$U_R\backslash\{A_1,\dots,A_m\}\cup\{B_1,\dots,B_m\}$.

\paragraph{Proiezione}%
\label{par:Proiezione}
La proiezione di una relazione $R$ su un insieme $A=\{A_1,\dots,A_m\}\subseteq
U_R$ di nomi di attributi di $R$, indicata con $\Pi_{A_1,\dots,A_m}\subseteq
(R)$, è una relazione di grado $m$ le cui tuple hanno come attributi solo gli
attributi specificati in $A$.
Pertanto la proiezione genera un insieme $T$ di tuple con $m$ attributi.
Sia $t=[A_1:v_1,\dots A_m:v_m]$ una tupla in $T$; $t$ è tale che esiste una
tupla $t^\prime$ in $R$ tale che $t[A]=t^\prime[A]$.
Nella relazione risultato gli attributi compaiono secondo l'ordine specificato
in $A$; è pertanto possibile specificare operazioni di proiezione che permutano
gli attributi di una relazione.
La proiezione permette di estrarre da una relazione solo alcune delle
informazioni in essa contenute, eliminando gli attributi al cui valore non siano
interessati.
La relazione risultato, oltre ad avere un grado inferiore a quello della
relazione argomento, può anche avere cardinalità inferiore a quella della
relazione argomento, poichè eliminando alcunti attributi possono venire generate
delle tuple duplicate che compariranno una sola volta nella relazione risultato.
