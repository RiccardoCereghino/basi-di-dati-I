\section{Algebra relazionale}%
\label{sec:Algebra relazionale}
L'algebra relazionale è costituita da $5$ operazioni per la manipolazione
delle relazioni:
\begin{itemize}
  \item proiezione;
  \item selezione;
  \item prodotto cartesiano;
  \item unione;
  \item differenza.
\end{itemize}

Ogni operazione ha come argomento una o più relazioni.
Esistono operazioni addizionali che possono essere espresse in termini delle
operazioni di base.

\subsection{Operazioni di base}%
\label{sub:Operazioni di base}
Sicchè facciamo riferimento alla notazione con nome, introduciamo l'operazione
di ridenominazione.

\paragraph{Ridenominazione}%
\label{par:Ridenominazione}
La ridenominazione di una relazione $R$ rispetto ad una lista di coppie di nomi
di attributi $(A_1,B_1),\dots,(A_m,B_m)$, tale che $A_i\in U_R$ è un nome di
attributo di $R$ indicata con $\rho_{A_1,\dots,A_m\leftarrow B_1,\dots B_m}(R)$,
ridenomina l'attributo di nome $A_i$ con il nome $B_i$, $i=1,\dots,m$.

La ridenominazione è corretta se il nuovo schema di relazione per $R$ ha
attributi con nomi tutti distinti.
La relazione ottenuta ha lo stesso grado della relazione $R$ ed ha lo stesso
contenuto.
Gli attributi della relazione risultato sono
$U_R\backslash\{A_1,\dots,A_m\}\cup\{B_1,\dots,B_m\}$.

\paragraph{Proiezione}%
\label{par:Proiezione}
La proiezione di una relazione $R$ su un insieme $A=\{A_1,\dots,A_m\}\subseteq
U_R$ di nomi di attributi di $R$, indicata con $\Pi_{A_1,\dots,A_m}\subseteq
(R)$, è una relazione di grado $m$ le cui tuple hanno come attributi solo gli
attributi specificati in $A$.
Pertanto la proiezione genera un insieme $T$ di tuple con $m$ attributi.
Sia $t=[A_1:v_1,\dots A_m:v_m]$ una tupla in $T$; $t$ è tale che esiste una
tupla $t^\prime$ in $R$ tale che $t[A]=t^\prime[A]$.
Nella relazione risultato gli attributi compaiono secondo l'ordine specificato
in $A$; è pertanto possibile specificare operazioni di proiezione che permutano
gli attributi di una relazione.
La proiezione permette di estrarre da una relazione solo alcune delle
informazioni in essa contenute, eliminando gli attributi al cui valore non siano
interessati.
La relazione risultato, oltre ad avere un grado inferiore a quello della
relazione argomento, può anche avere cardinalità inferiore a quella della
relazione argomento, poichè eliminando alcunti attributi possono venire generate
delle tuple duplicate che compariranno una sola volta nella relazione risultato.

\paragraph{Selezione}%
\label{par:Selezione}
La selezione su una relazione $R$, dato un predicato $F$ su $R$, indicata con
$\sigma_F(R)$, genera una relazione che contiene tutte le tuple di $R$ che
verificano $F$.
Un predicato $F$ su $R$ è un predicato semplice su $R$, oppure una combinazione
booleana, ottenuta mediante gli operatori booleani $\wedge$, $\vee$, $\neg$, di
predicati semplici su $R$.
Un predicato semplice su $R$ i cui domini devono essere compatibili; $op$ è un
operatore relazionale di confronto ed appartiene all'insieme
$\{<,=,>,\geq,\leq\}$; $v$ è un valore costante compatibile con il dominio
$A$.

Se il grado della relazione operando $R$ è $k$, la selezione $\sigma_F(R)$
genera un insieme $T$ di tuple di grado $k$ (sottoinsieme di $R$). Quindi lo
scema (ed il grado) della relazione risultato sono uguali a quelli della
relaione operando.

\paragraph{Prodotto cartesiano}%
\label{par:Prodotto cartesiano}
Il prodotto cartesiano di due relazioni $R$ ed $S$, di grado rispettivamente
$k_1$ e $k_2$, indicato con $R\times S$, è una relazione di grado $k_1+k_2$ le
cui tuple sono tutte le possibili tuple ch hanno:
\begin{itemize}
  \item come prime $k_1$ componenti tuple di $R$;
  \item come ultime $k_2$ componenti tuple di $S$.
\end{itemize}

Il prodotto cartesiano può essere applicato solo se le due relazioni $R$ ed $S$
hanno schemi disgiunti. Nella relazione risultato i primi $k_1$attributi sono
gli attributi della relazione $R$, gli ultimi $k_2$ attributi sono gli attributi
della relazione $S$.
Il prodotto cartesiano permette di costruire nuove tuple combinando le
informazioni presenti nelle tuple delle relazioni argomento.
Poichè ogni tupla di $R$ viene combinata con ogni tupla di $S$, la cardinalità
del risultato è il ptodotto delle cardinalità degli argomenti.

\paragraph{Unione}%
\label{par:Unione}
L'unione delle relazioni $R$ ed $S$, indicata con $R\cup S$, è l'insieme delle
tuple che sono in $R$ ed in $S$.
L'unione delle relazioni può essere eseguita solo se le relazioni hanno lo
stesso schema.
La relazione risultato ha lo stesso schema delle relazioni argomento.
Essendo l'unione un'operazione tra insiemi, le tuple duplicate vengono eliminate
dal risultato.

\paragraph{Differenza}%
\label{par:Differenza}
La differenza delle relazioni $R$ ed $S$, indicata con $R-S$, è l'insieme delle
tuple che sono in $R$ ma non in $S$.
La differenza, come l'unione, di due relazioni può essere eseguita solo se le
relazioni hanno lo stesso schema e produce una relazione con lo stesso schema.
Se le relazioni hanno attributi con nomi diversi, si applica quanto già detto
per l'unione.

\subsection{Operazioni derivate}%
\label{sub:Operazioni derivate}
Le operazioni derivate sono definite attraverso le operazioni di base.

\paragraph{Intersezione}%
\label{par:Intersezione}
L'intersezione di due relazioni $R$ ed $S$ è denotata con $R\cap S$ ed è
espressa come $R-(R-S)$.
L'intersezione di $R$ ed $S$ restituisce le tuple che sono sia in $R$ che in
$S$. L'intersezione può essere eseguita solo se le relazioni hanno lo stesso
schema e produuce una relazione con lo stesso schema.

\paragraph{Join}%
\label{par:Join}
Il join (detto anche \textit{theta-join}) di due relazioni $R$ ed $S$ sugli
attributi $A$ di $R$ ed $A^\prime$ di $S$, indicato con $R^{\bowtie}_{A\Theta
A^\prime}S$, dove $\Theta$ è un operatore relazionale di confronto, è definito
dall'espressione algebrica $\sigma_{A\Theta A^\prime}(R\times S)$.
Il join è perdanto un prodotto cartesiano seguito da una selezione; il predicato
$A\Theta A^\prime$ è detto predicato di join.
Come per il prodotto cartesiano, gli schemi delle due relazioni argomento devono
essere disgiunti e lo schema della relazione risultato è dato dalla loro unione.

Il join permette di collegare tuple di relazioni diverse e di attraversare le
associazioni rappresentate nelle relazioni della base di dati mediante il
meccanismo delle chiavi esterne.

\paragraph{Join naturale}%
\label{par:Join naturale}
Rappresenta una semplificazione del join.
Il join naturale di dure relazioni $R$ ed $S$, denotato come $R\bowtie S$, è
definito come:

sia $\{A_1,\dots,A_k\}=U_R\cap U_S$ l'insieme dei nomi di attributi presenti sia
nello schema di $R$ sia in quello di $S$.
Sia inoltre $\{I_1,\dots,I_k\}=U_R\cap U_S$ l'insieme dei nomi di attributo
unione dell'insieme dei nomi degli attributi di $R$ e dell'insieme dei nomi
degli attributi di $S$.
Siano infine $\{B_1,\dots,B_k\}$ nomi di attributo non appartenenti nè ad $R$ nè
ad $S$.

L'espressione che definisce il join naturale è:
\[
  R\bowtie S=\Pi_{I_1,\dots,I_m}(\sigma_C(R\times(\rho_{A_1,\dots,A_k\leftarrow
  B_1,\dots,B_k}(S))))
\]

dove $C$ è un predicato della forma $A_1=B_1\wedge A_2=B_2\wedge\dots\wedge
A_k=B_k$.

Pertanto il join naturale esegue un join eguagliando gli attributi con lo stesso
nome delle due relazioni argomento dell'operazione e poi elimina gli attributi
duplicati dalla relazione risultato.
Si noti che, affinchè l'operazione di join sia ben definita, gli attributi con
nomi uguali nelle relazioni argomento dell'operazione devono avere domini
compatibili.
