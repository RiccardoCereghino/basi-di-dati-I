\chapter{DBMS - Data Base Managament System}%
\label{cha:DBMS - Data Base Managament System}
\section{Sistema di gestione di basi di dati}%
\label{sec:Sistema di gestione di basi di dati}
I sistemi di gestione di basi di dati si pongono di risolvere i problemi legati
alla gestione e salvataggio di dati, quali:
\begin{itemize}
  \item \textbf{ridondanza ed inconsistenza dei dati}, ovvero la duplicazione
    dei dati su file multipli, che comporta anche un pericolo di inconsistenza;
  \item \textbf{difficoltà nell'accesso ai dati}, la mancanza di una descrizione
    di alto livello e centralizzata dei dati ne rende estremamente difficoltoso
    l'utilizzo al fine di rispondere a nuove esigenze applicative, a questo
    proposito i \emph{DBMS} forniscono linguaggi per facilitare l'accesso ai
    dati, integrazione con i linguaggi di programmazione e funzionalità
    reattive;
  \item \textbf{problemi nell'accesso concorrente ai dati}, operazioni multiple
    sullo stesso dato possono causare problemi di concorrenza, i \emph{DBMS}
    mettono a disposizione il concetto di \textit{transazione}, che garantisce
    la consistenza dei dati in presenza di transazioni concorrenti;
  \item \textbf{problemi di integrità dei dati}, i \emph{DBMS} forniscono la
    possibilità di stabilire dei vincoli al salvataggio ed alla modifica dei
    dati, preservandone l'integrità.
\end{itemize}

\subsection{Obiettivi e servizi di un DBMS}%
\label{sub:Obiettivi e servizi di un DBMS}
I DBMS implementano una serie di servizi per offrire l'utilizzo della base di
dati e lo sviluppo di applicazioni con cui si interfacciano, alcuni sono
invocabili dagli utenti, altri sono utilizzati per il funzionamento interno.

I DBMS adottano un'\textit{architettura client-server}, per cui le funzionalità
del sistema sono realizzate da due moduli distinti: il \textbf{client} che
gestisce l'interazione tra utente e DBMS ed il \textbf{server} che si occupa
della memorizzazione e gestione dei dati.

\subsubsection{Principali servizi offerti da un DBMS}%
\label{ssub:Principali servizi offerti da un DBMS}
\begin{itemize}
  \item \textbf{Descrizione dei dati:} per specificare i dati da memorizzare
    nella base di dati;
  \item \textbf{manipolazione dei dati, per:}
    \begin{itemize}
      \item accedere ai dati;
      \item inserire nuovi dati;
      \item modificare dati esistenti;
      \item cancellare dati esistenti;
    \end{itemize}
  \item \textbf{controllo di integrità:} per evitare di memorizzare dati non
    corretti;
  \item \textbf{strutture di memorizzazione:} per rappresentare in memoria
    secondaria i costrutti del modello dei dati;
  \item \textbf{ottimizzazione di interrogazioni:} per determinare la strategia
    più efficiente per accedere ai dati;
  \item \textbf{protezione dei dati:} per proteggere i dati da accessi non
    autorizzati;
  \item \textbf{ripristino della base di dati}, per evitare che errori e
    malfunzionamenti:
    \begin{itemize}
      \item determinino una base di dati inconsistente;
      \item provochino perdite di dati;
    \end{itemize}
  \item \textbf{controllo della concorrenza:} per evitare che accessi
    concorrenti alla base di dati provochino inconsistenze dei dati.
\end{itemize}

\subsection{Modelli dei dati}%
\label{sub:Modelli dei dati}
Una delle caratteristiche principali di un DBMS è di poter disporre di una
rappresentazione logica ad alto livello dei dati, i \emph{modelli di dati}
astraggono i dati presenti nel sistema per essere utilizzati più semplicemente
da utenti ed applicazioni.

\subsubsection{Concetti di base}%
\label{ssub:Concetti di base}
Un \emph{modello di dati} è un formalismo che racchiude tre componenti
fondamentali:
\begin{itemize}
  \item un insieme di strutture dati;
  \item un linguaggio per specificare, aggiornare e vincolare le strutture dati
    previste dal modello;
  \item un linguaggio per manipolare i dati.
\end{itemize}

\paragraph{Struttura dati}%
\label{par:Struttura dati}
Ad modello di dati corrisponde una struttura dati, composta da:
\begin{itemize}
  \item \textbf{entità:} insieme di oggetti della realtà applicativa di
    interesse, aventi caratteristiche comuni;
  \item \textbf{istanza di entità:} singolo oggetto della realtà applicativa di
    interesse, modellato da una certa entità;
  \item \textbf{attributo:} proprietà significativa di un entità, ai fini della
    descrizione della realtà applicativa di interesse (ogni entità è
    caratterizzata da uno o più attributi), un attributo di un entità assume uno
    o più valori per ciascuna delle istanze dell'entità, detti \textit{valori
    dell'attributo}, in un insieme di possibili valori, detto \textit{dominio
    dell'attributo};
  \item \textbf{associazione:} corrispondenza tra un certo numero di entità,
    anche le associazioni possono avere degli attributi che corrispondono alle
    proprietà delle associazioni;
  \item \textbf{istanza di associazione:} corrispondenza tra le istanze di un
    certo numero di entità.
\end{itemize}

\subsection{Modello relazionale}%
\label{par:Modello relazionale}
Il modello relazionale è basato su di una singola struttura dati, la relazione.
Una relazione viene solitamente rappresentata come una tabella con righe (dette
tuple) e colonne contenenti dati di tipo specificato.

Una tupla tipicamente rappresenta un'istanza dell'entità modellata dalla tabella
a cui la tupla appartiene, mentre la colonna di una tabella rappresenta un
particolare attributo dell'entità modellata dalla tabella stessa.

Attributi con la proprietà di identificare univocamente le tuple di una
relazione vengono chiamati \textit{chiavi}, mentre i corrispondenti attributi
nell'altra relazione, prendono il nome di \textit{chiavi esterne}.

\subsection{Schemi ed istanze}%
\label{ssub:Schemi ed istanze}
In un DBMS individuiamo lo \textit{schema della base di dati} e l'
\textit{istanza della base di dati}.

Lo schema fornisce una descrizione dei dati per tipo, l'istanza è l'insieme
delle tuple contenute nella base di dati.

\subsection{Livelli nella rappresentazione dei dati}%
\label{ssub:Livelli nella rappresentazione dei dati}
Uno degli scopi di un DBMS è di fornire una rappresentazione ad alto livello di
un insieme di dati nascondendone l'effettiva memorizzazione, per cui la base di
dati fornisce tre livelli diversi di visualizzazione/astrazione:
\begin{itemize}
  \item \textbf{livello fisico:} il livello più basso con cui viene definito lo
    schema fisico della base di dati, precisando come i dati sono effettivamente
    memorizzati tramite strutture di memorizzazione;
  \item \textbf{livello logico:} il secondo livello di astrazione in cui viene
    decritto lo \textit{schema logico}, ovvero quali sono i dati memorizzati
    nella base di dati, eventuali associazione tra di essi ed eventuali vincoli
    di integrità semantica e di autorizzazione, l'intera base di dati è
    descritta tramite un numero limitato di strutture dati che costituiscono il
    modello dei dati;
  \item \textbf{livello esterno:} è il livello di astrazione più alto, descrive
    una porzione dell'intero schema della base di dati, possono essere definite
    più viste di una stessa base di dati.
\end{itemize}

La presenza di questi livelli di astrazione assicura alcune importanti proprietà
ai datim l' \textbf{indipendenza fisica} e l' \textbf{indipendenza logica}.

Il concetto di indipendenza fisica esprime la possibilità di interagire con il
livello logico senza apportare modifiche al livello fisico.

Il concetto di indipendenza logica consente la possibilità di interagire con il
livello esterno senza apportare modifiche al livello logico.

\subsection{Linguaggi di un DBMS}%
\label{ssub:Linguaggi di un DBMS}
I DBMS mettono a disposizione un insieme di linguaggi che permettono agli utenti
di interagire con il DBMS per descrivere e manipolare i dati di interesse e
specificare vincoli, tra questi i linguaggi più rilevanti sono:

\paragraph{DDL - Data Definition Language}%
\label{par:DDL - Data Definition Language}
Il linguaggio di definizione dei dati consente di specificare ed aggiornare lo
schema di una base di dati, in particolare il DDL concretizza il modello dei
dati fornendo la notazione che permette di specificarne le strutture dati.

Il DDL deve supportare la specifica del nome della base di dati, come pure di
tutte le unità logiche elementari della base di dati e di eventuali vincoli di
integrità semantica e di autorizzazione prevedendo comandi per l'aggiornamento
delle strutture dati previste dal modello.

Per quanto concerne i linguaggio di manipolazione dei dati si distingue tra
\emph{linguaggi procedurali (operazionali)} e \emph{linguaggi non procedurali
(dichiarativi)} per esempio \textbf{SQL}.

\paragraph{DML - Data Manipulation Language}%
\label{par:DML - Data Manipulation Language}
Una base di dati, organizzata logicamente tramite il modello dei dati e definita
tramite il DDL è accessibile agli utenti ed alle applicazioni tramite il DML.
Le operazioni fornite da questo linguaggio servono per gestire le istanze della
base di dati e sono fondamentalmente quattro:
\begin{itemize}
  \item \textbf{inserimento:} per l'immisione di nuovi dati;
  \item \textbf{ricerca:} per il ritrovamento dei dati in interesse, detta
    interrogazione (\textit{query});
  \item \textbf{cancellazione:} per l'eliminazione di dati obsoleti;
  \item \textbf{modifica:} per variare dati esistenti.
\end{itemize}

\paragraph{SDL - Storage Definition Language}%
\label{par:SDL - Storage Definition Language}
La corrispondenza tra le strutture logiche e le strutture di memorizzazione deve
essere opportunamente definita.
Nella maggior parte dei DBMS attuali la definizione di tale corrispondenza è
eseguita automaticamente dal DBMS stesso una volta che lo schema logico è
definito, tuttavia l'utente può influenzare le scelte operate dal DBMS tramite i
comandi del linguaggio di definizione delle strutture di memorizzazione.
