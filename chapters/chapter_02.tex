\chapter{Modello relazionale}%
\label{cha:Modello relazionale}

\section{Modello dei dati}%
\label{sec:Modello dei dati}
Il modello relazionale è stato proposto da \emph{E.F. CODD} nel 1970.

Le interrogazioni sulle relazioni possono essere espresse in due formalismi:
\begin{itemize}
  \item \textbf{algebra relazionale:} in cui le interrogazioni sono espresse
    applicando operatori specializzati alle relazioni;
  \item \textbf{calcolo relazionale:} in cui le interrogazioni sono espresse per
    mezzo di formule logiche che devono essere verificate dalle tuple ottenute
    come risposta all'interrogazione.
\end{itemize}

\subsection{Relazioni}%
\label{sub:Relazioni}
Il concetto alla base del modello relazionale è la \textit{relazione}, definita
partendo dalla nozione di dominio, un dominio è un insieme anche infinito di
valori.

Sia $\D$ l'insieme di tutti i domini, che assumeremo contenere i tipi:
\begin{itemize}
  \item \textbf{int:} numeri interi;
  \item \textbf{real:} numeri reali;
  \item \textbf{string:} stringhe;
  \item \textbf{date:} date.
\end{itemize}

I valori dei domini costituiscono i valori atomici che popleranno la base di
dati, a partire dai quali vengono costruite le \emph{tuple} delle relazioni,
effeuando un prodotto cartesiano di tali valori.

\begin{definizione}[Prodotto cartesiano]
Siano $D_1,D_2,\dots,D_k\in\D$ \emph{k} domini.
Il prodotto cartesiano di tali domini, indicato con $D_1\times
D_2\times\dots\times D_k$ è definito come l'insieme:
\[
  \{(v_1,v_2,\dots,v_k)|v_1\in\D_1,\dots,v_k\in\D_k\}
\]
Gli elementi appartenenti al prodotto cartesiano sono detti \textbf{tuple}; il
prodotto cartesiano di \emph{k} domini ha grado \emph{k}.
\end{definizione}

\begin{definizione}[Relazione]
Siano $D_1,D_2,\dots,D_k\in\D$ \emph{k} domini.
Una relazione su $D_1,D_2,\dots,D_k\in\D$ è un sottoinsieme finito del prodotto
cartesiano $D_1\times D_2\times\dots\times D_k$.
\end{definizione}

\subsubsection{Nomenclatura}%
\label{ssub:Nomenclatura}
\begin{itemize}
  \item \textbf{Relazione:} sottonsieme del prodotto cartesiano di $k$ domini,
    ha grado $k$;
  \item \textbf{tupla:} ogni tupla di una relazione di grado $k$ ha $k$
    componenti, uno per ogni dominio su cui è definita la relazione cui la tupla
    appartiene.
  \item \textbf{componente:} data una relazione $R$ di grado $k$, una tupla
    $t\in\R$ ed un intero $i\in\{1,\dots,k\}$, la notazione $t[i]$ denota la
    $i$-esima componente di $t$;
  \item \textbf{cardinalità:} esprime il numero di tuple appartenenti alla
    relazione, una relazione è sempre un insieme finito;
  \item \textbf{attributo:} un'alternativa alla formulazione del modello
    relazionale è di associare un nome, detto \emph{nome di attributo} ad ogni
    componente delle tuple in una relazione.
\end{itemize}

\begin{definizione}[Schema di relazione]
  Siano $R$ un nome di relazione, $\{A_1,\dots,A_n\}$ un insieme di nomi di
  attributi, $\dom:\{A_1,\dots,A_n\}\rightarrow\D$ una funzione totale che
  associa ad ogni nome di attributo in $\{A_1,\dots,A_n\}$ il corrispondente
  dominio.

  La coppia $(R(A_1,\dots,A_n),\dom)$ è uno schema di relazione.
  $U_R$ denota l'insieme dei nomi di attributi di $R$ cioè
  $\{\{A_1,\dots,A_n\}\}$.
\end{definizione}

\begin{definizione}[Schema di base di dati]
Siano $S_1,\dots,S_n$ schemi di relazioni, con nomi di relazione diversi,
$S=\{S_1,\dots,S_n\}$ è detto schema di base di dati.
\end{definizione}

\begin{definizione}[Tupla e relazione]
Sia $R(A_1,\dots,A_n),\dom)$ uno schema di relazione.

Una tupla $t$ definita su $R$ è un insieme di funzioni totali $f_1,\dots,f_n$,
dove $f_i:A_i\rightarrow\dom)A_i),i=1,\dots,n$, associa all'attributo di nome
$A_i$ un valore del dominio di tale attributo.
Una relazione definita su uno schema di relazione è un insieme finito di tuple
definite su tale schema; tale relazione è anche detta \emph{istanza dello
schema}.
\end{definizione}

\subsection{Valori nulli}%
\label{sub:Valori nulli}
Il valore nullo è un valore ammissibile per ogni dominio, rappresenta la
mancanza di un valore di una tupla.

Denotiamo il valore nullo con il simbolo $?$.
