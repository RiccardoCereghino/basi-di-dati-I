\section{Architettura di un DBMS}%
\label{sec:Architettura di un DBMS}
Le basi di dati memorizzano in modo persistente grosse quantità di dati,
memorizzate in memoria secondaria; per cui non possono essere elaborate
direttamente dalla CPU, ma copiati in memoria principali in un \emph{buffer}.

Quindi un DBMS è costituito da diverse componenti funzionali che includono:
\begin{itemize}
  \item \textbf{gestore dei file:} gestisce l'allocazione dello spazio su disco
    e le strutture dati usate per rappresentare le informazioni memorizzate su
    disco;
  \item \textbf{gestore del buffer:} responsabile del trasferimento delle
    informazioni tra disco e memoria principale;
  \item \textbf{esecutore di interrogazioni:} responsabile dell'esecuzione delle
    richieste utente e costituito da:
    \begin{itemize}
      \item \textbf{parser:} traduce i comandi del DDL e del DML in un formato
        interno (\textit{parse tree});
      \item \textbf{selezionatore del piano:} stabilisce il modo più efficiente
        di processare una richiesta utente;
      \item \textbf{esecutore del piano:} processa le richieste utente in
        accordo al piano di esecuzione selezionato;
    \end{itemize}
  \item \textbf{gestore delle autorizzazioni:} controlla che gli utenti abbiano
    gli opportuni diritti di accesso ai dati;
  \item \textbf{gestore del ripristino:} assicura che la base di dati rimanga in
    uno stato consistente a fronte di cadute o malfunzionamenti del sistema;
  \item \textbf{gestore della concorrenza:} assicura che le esecuzioni
    concorrenti di processi procedano senza conflitti.
\end{itemize}

Quindi il DBMS memorizza, oltre ai dati utente, anche dati di sistema, strutture
ausiliarie di accesso a tali dati e dati statistici, utilizzati dal
selezionatore del piano per determinare la migliore strategia di esecuzione.

\subsection{Cataloghi di sistema}%
\label{sub:Cataloghi di sistema}
Un DBMS sdescrive i dati che gestisce, incluse le informazioni sullo schema
della base di dati e gli indici, tramite \emph{meta-dati} che, sono memorizzate
in relazioni speciali dette \emph{cataloghi di sistema}, utilizzati dalle
diverse componenti del DBMS.

Un aspetto elegante di un DBMS relazionale è che i cataloghi di sistema sono
essi stessi relazioni.
Anche i cataloghi di sistema permettono l'interrogazione e le techiche per
l'implementazione e la gestione di relazioni.
