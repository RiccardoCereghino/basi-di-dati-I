\chapter{Memorizzazione dei dati ed elaborazione delle interrogazioni}%
\label{cha:Memorizzazione dei dati ed elaborazione delle interrogazioni}
Le prestazioni di un DBMS dipendono dall'efficienza delle strutture dati e
dall'efficienza del sistema nell'operare su tali strutture.
A livelloi fisico le tuple sono memorizzate in record di file su memoria
secondaria ed una manipolazione efficiente verrà garantita dall'uso di opportune
tecniche di elaborazione delle interrogazioni.

Per velocizzare la ricerca dei dati vengono in genere utilizzate particolari
strutture di accesso, dette \emph{indici}, che consentono di accedere
direttamente ai record corrispondenti alle tuple con un certo valore per un
attributo, senza scandire l'intero contenuto del file.

La scelta delle strutture di memorizzazione e di indicizzazione più efficienti
dipende dal tipo di accessi che si eseguono sui dati; ogni DBMS ha le proprie
strategie di implementazione di un modello di dati e le scelte dell'utente a
questo proposito costituiscono la \emph{progettazione fisica della base di
dati}.

\subimport{./chapter-07/}{section-01.tex}
