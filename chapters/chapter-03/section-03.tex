\section{Funzioni di gruppo}%
\label{sec:Funzioni di gruppo}
La possibilità di estrarre tuple di una relazione è data dalle funzioni di
gruppo e dall'operatore di raggruppamento.

\subsection{Funzioni di gruppo}%
\label{sub:Funzioni di gruppo}
Le funzioni di gruppo, o \emph{funzioni aggregate}, possono essere utilizzate
nella clausola di proiezione di un interrogazione e permettono di estrarre
informazioni riassuntive da insiemi di valori.

Una funzione di gruppo si applica all'iinsieme dei valori estratti dalle tuple
che soddisfano la clausola di qualificazione dell'interrogazioni.
Le principali funzioni di gruppo previste da SQL sono:
\begin{itemize}
  \item \lstinline[language=SQL]{MAX}, determina il massimo in un insieme di
    valori;
  \item \lstinline[language=SQL]{MIN}, determina il minimo in un insieme di
    valori;
  \item \lstinline[language=SQL]{SUM}, esegue la somma dei valori in un insieme;
  \item \lstinline[language=SQL]{AVG}, esegue la media dei valori in un insieme;
  \item \lstinline[language=SQL]{COUNT}, determina la cardinalità di un insieme:
  \item \lstinline[language=SQL]{STDEV}, esegue il calcolo della deviazione
    standard;
  \item \lstinline[language=SQL]{VAR}, esegue il calcolo della varianza.
\end{itemize}

Le funzioni \lstinline[language=SQL]{SUM,AVG,COUNT,STDEV} sono
applicate solo per insiemi di valori numerici, mentre le altre possono essere
applicate anche ad altri insiemi di valori.

Ad eccezione della funzione \lstinline[language=SQL]{COUNT}, le funzioni devono
operare solo su insiemi di valori semplici, mentre
\lstinline[language=SQL]{COUNT} prende come argomenti:
\begin{enumerate}
  \item \textbf{un nome di colonna:} in tal caso la funzione restituisce il
    numero di valori non nulli presenti nella colonna;
  \item \textbf{un nome di colonna preceduto dal qualificatore DISTINCT:} in tal
    caso la funzione restituisce il numero di valori distinti e non nulli
    presenti nella colonna;
  \item \textbf{il carattere speciale $*$:} in tal caso la funzione restituisce
    il numero di tuple.
\end{enumerate}

Le colonne della relazione risultato ottenute dall'applicazione di funzioni di
gruppo sono colonne virtuali; per default verrà assegnato il nome della funzione
usata nel calcolo della colonna.

\subsection{Raggruppamento}%
\label{sub:Raggruppamento}
L'operatore di raggruppamento permette di suddividere le tuple di una relazione
in base al valore di una o più colonne di ttale relazione. Le colonne da usare
nel partizionamento sono specificate in un'apposita clausola
\lstinline[language=SQL]{GROUP BY} del comando \lstinline[language=SQL]{SELECT}.

Le tuple su cui si esegue il partizionamento sono solo le tuple che verificano
la clausola di qualificazione del comando \lstinline[language=SQL]{SELECT}.
Le tuple che non verificano tale clausola non partecipano al partizionamento.

Il risultato di un comando \lstinline[language=SQL]{SELECT} che contiene la
clausola di raggruppamento contiene tante tuple quanti sono i gruppi di tuple
risultanti dal partizionamento.
La clausola di proiezione di un'interrogazione contenente una clausola
\lstinline[language=SQL]{GROUP BY} può includere solamente colonne che compaiono
nella clausola \lstinline[language=SQL]{GROUP BY} oppure nelle funzioni di
gruppo. Non è invece possibile includere in tale clausola colonne delle
relazioni che non compaiono nella clausola \lstinline[language=SQL]{GROUP BY}.

Una volta raggruppate le tuple, infatti, si perde la possibilità di riferire il
valore di tali colonne delle tuple originarie: il risultato contiene una sola
tupla per ogni gruppo mentre le singole colonne delle tuple nel gruppo possono
contenere diversi valori.

Il raggruppamento piò essere applicato a relazioni ottenute come risultato di
operazioni di join. In particolare, la clausola 
\lstinline[language=SQL]{GROUP BY} può avere come argomenti colonne di relazioni
diverse, tra quelle che partecipano all'operazione di join.

SQL prevede inoltre la possibilità di specificare condizioni di ricerca su
gruppo di tuple. Queste condizioni di ricerca permettono di scartare alcuni dei
gruppi ricavati dal partizionamento ottenuto in base alla clausola
\lstinline[language=SQL]{GROUP BY}.
Condizioni di ricerca su gruppi di tuple sono specificate nell'apposita clausola
\lstinline[language=SQL]{HAVING}, la condizione di ricerca può essere una
combinazione booleana di più predicati.

Tali predicati possono essere solo predicati che coinvolgono funzioni di gruppo.
La clausola \lstinline[language=SQL]{HAVING}, infatti, non può contenere
condizioni sui valori delle colonne delle singole tuple.

\subsection{Valori nulli}%
\label{sub:Valori nulli}
Una o più tuple possono, in un interrogazione, contenere valori nulli, per cui
SQL utilizza una logica booleana di tre valori che sono:
\lstinline[language=SQL]{TRUE(T),FALSE(F),UKNOWN(?)}.

Nelle espressioni se un argomento è \lstinline[language=SQL]{NULL}, il valore
dell'intera espressione sarà \lstinline[language=SQL]{NULL}, mentre in una
funzione di gruppo tali valori verranno esclusi.

\subsection{Sotto-interrogazioni}%
\label{sub:Sotto-interrogazioni}
La clausola \lstinline[language=SQL]{WHERE}, detta inerrogazione esterna, può
contenere un'altra interrogazione, detta sotto-interrogazione
(\textit{subquery}).

Ogni sotto-interrogazione restituisce un solo valore, dette sotto-interrogazioni
scalari; se venisse restituita più di una tupla verrebbe generato un errore a
run-time.

Se nessun valore venisse restituito verrebbe tornato
\lstinline[language=SQL]{NULL}.

Se più di un valore venisse restituite è necessario specificare come tali valori
devono essere usati nel predicato, per cui sono introdotti i quantificatori
\lstinline[language=SQL]{ANY,ALL} che sono inseriti tra l'operatore di confronto
e la sotto-interrogazione.

\lstinline[language=SQL]{ANY} restituisce il valore di verità $T$ quando la
valutazione di confronto restituisce $T$ su almeno una delle tuple e $F$ se
nessuna tupla viene restituita.

\lstinline[language=SQL]{ALL} restituisce $T$ se tutte le tuple restituiscono
$T$ e se nessuna tupla viene restituita restituisce $T$.

Si può selezionare più di una colonna tramite una sotto-interrogazione e una
sotto-interrogazione può includere nella propria condizione di ricerca un'altra
sotto-interrogazione.
