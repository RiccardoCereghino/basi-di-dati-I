\section{Linguaggio di definizione dei dati}%
\label{sec:Linguaggio di definizione dei dati}
Le relazioni possono essere definite tramite il comando $CREATE$, modificate
tramite il comando $ALTER$ e cancellate tramite il comando $DROP$.
Questi comandi sono i principali del \emph{DDL}.

\subsection{Tipi di dato}%
\label{sub:Tipi di dato}
Vi sono quattro diverse categorie di tipi di dato, divisi a loro volta per
sottocategorie:
\begin{itemize}
  \item tipi numerici;
  \item tipi carattere;
  \item tipi temporali;
  \item tipi definiti dall'utente.
\end{itemize}

\subsubsection{Tipi numerici}%
\label{ssub:Tipi numerici}
Sono classificati in \textit{tipi numerici esatti} e \textit{tipi numerici
approssimati}.

\paragraph{Tipi numerici esatti}%
\label{par:Tipi numerici esatti}
\begin{itemize}
  \item \textbf{INTEGER:} rappresenta i valori interi, la precisione varia a
    seconda della specifica implementazione di SQL;
  \item \textbf{SMALLINT:} rappresenta i valori interi, deve essere meno preciso
    di INTEGER;
  \item \textbf{BIGINT:} rappresenta i valori interi, deve essere più preciso di
    INTEGER;
  \item \textbf{NUMERIC:} tipo di dato caratterizzato da una precisione (numero
    totale di cifre) e da una scala (numero di cifre dopo la virgola decimale),
    la specifica ha forma $NUMERIC[(p[,s])]$, dove i predefiniti sono $p=1$ e
    $s=0$;
  \item \textbf{DECIMAL:} analogo a $NUMERIC$, differisce nella possibilità di
    inserire numeri meno precisi rispetto a quanto definito.
\end{itemize}

\paragraph{Tipi numerici}%
\label{par:Tipi numerici}
\begin{itemize}
  \item \textbf{REAL:} rappresenta valori a singola precisione in virgola
    mobile, la precisione varia a seconda della specifica implementazione di
    SQL;
  \item \textbf{DOUBLE PRECISION:} rappresenta valori a singola precisione in
    virgola mobile (solitamente a doppia precisione), più preciso rispetto a
    $REAL$;
  \item \textbf{FLOAT:} permette di richiedere la precisione desiderata, ha
    forma $FLOAT[(p)]$, la precisione minima è $1$, la precisione di default e
    la massima variano a seconda dell'implementazione di SQL.
\end{itemize}

\paragraph{Tipi di carattere}%
\label{par:Tipi di carattere}
\begin{itemize}
  \item \textbf{CHARACTER:} permette definire stringhe di caratteri di lunghezza
    predefinita nella forma $CHAR(n)$, la stringa viene completata con spazi
    vuoti, il valore di default di $n$ è $1$;
  \item \textbf{CHARACTER VARYING:} permette di definire stringhe di caratteri
    di una lunghezza massima predefinita, di forma $VARCHAR(n)$, differisce da
    $CHAR$ nella possibilità di non dover occupare tutto lo spazio allocato.
\end{itemize}

\paragraph{Tipi temporali}%
\label{par:Tipi temporali}
\begin{itemize}
  \item \textbf{DATE:} rappresenta le date espresse come anno ($4$ cifre), mese
    ($2$ cifre) e giorno ($2$ cifre), sono disponibili diversi formati di
    rappresentazione;
  \item \textbf{TIME:} rappresenta i tempi espressi come ora ($2$ cifre), minuto
    ($2$ cifre) e secondo ($2$ cifre), la specifica ha forma $TIME[(p)]$ dove
    $p$ è l'eventuale numero di cifre frazionarie cui si è interessati ($6$
    cifre, microsecondo);
  \item \textbf{TIMESTAMP:} rappresenta una concatenazione dei tipi di dato
    $DATE$ e $TIME$, la specifica è $TIMESTAMP[(p)]$;
  \item \textbf{INTERVAL:} rappresenta una durata temporale in riferimento ad
    uno o più qualificatori tra $YEAR$, $MONTH$, $DAY$, $HOUR$, $MINUTE$ e
    $SECOND$, i valori di questo tipo sono rappresentati dalla parola chiave
    $INTERVAL$ seguita da una stringa che caratterizza la durata in termini di
    uno o più qualificatori.
\end{itemize}

\paragraph{Altri tipi definiti}%
\label{par:Altri tipi definiti}
\begin{itemize}
  \item \textbf{BOOLEAN:} i cui valori sono $TRUE$, $FALSE$, $UNKOWN$;
  \item \textbf{BLOB:} \emph{Binary Large OBject};
  \item \textbf{CLOB:} \emph{Character Large Object}.
\end{itemize}
