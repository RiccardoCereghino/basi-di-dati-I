\section{Linguaggio di definizione dei dati}%
\label{sec:Linguaggio di definizione dei dati}
Le relazioni possono essere definite tramite il comando $CREATE$, modificate
tramite il comando $ALTER$ e cancellate tramite il comando $DROP$.
Questi comandi sono i principali del \emph{DDL}.

\subsection{Tipi di dato}%
\label{sub:Tipi di dato}
Vi sono quattro diverse categorie di tipi di dato, divisi a loro volta per
sottocategorie:
\begin{itemize}
  \item tipi numerici;
  \item tipi carattere;
  \item tipi temporali;
  \item tipi definiti dall'utente.
\end{itemize}

\subsubsection{Tipi numerici}%
\label{ssub:Tipi numerici}
Sono classificati in \textit{tipi numerici esatti} e \textit{tipi numerici
approssimati}.

\paragraph{Tipi numerici esatti}%
\label{par:Tipi numerici esatti}
\begin{itemize}
  \item \textbf{INTEGER:} rappresenta i valori interi, la precisione varia a
    seconda della specifica implementazione di SQL;
  \item \textbf{SMALLINT:} rappresenta i valori interi, deve essere meno preciso
    di INTEGER;
  \item \textbf{BIGINT:} rappresenta i valori interi, deve essere più preciso di
    INTEGER;
  \item \textbf{NUMERIC:} tipo di dato caratterizzato da una precisione (numero
    totale di cifre) e da una scala (numero di cifre dopo la virgola decimale),
    la specifica ha forma $NUMERIC[(p[,s])]$, dove i predefiniti sono $p=1$ e
    $s=0$;
  \item \textbf{DECIMAL:} analogo a $NUMERIC$, differisce nella possibilità di
    inserire numeri meno precisi rispetto a quanto definito.
\end{itemize}

\paragraph{Tipi numerici}%
\label{par:Tipi numerici}
\begin{itemize}
  \item \textbf{REAL:} rappresenta valori a singola precisione in virgola
    mobile, la precisione varia a seconda della specifica implementazione di
    SQL;
  \item \textbf{DOUBLE PRECISION:} rappresenta valori a singola precisione in
    virgola mobile (solitamente a doppia precisione), più preciso rispetto a
    $REAL$;
  \item \textbf{FLOAT:} permette di richiedere la precisione desiderata, ha
    forma $FLOAT[(p)]$, la precisione minima è $1$, la precisione di default e
    la massima variano a seconda dell'implementazione di SQL.
\end{itemize}

\paragraph{Tipi di carattere}%
\label{par:Tipi di carattere}
\begin{itemize}
  \item \textbf{CHARACTER:} permette definire stringhe di caratteri di lunghezza
    predefinita nella forma $CHAR(n)$, la stringa viene completata con spazi
    vuoti, il valore di default di $n$ è $1$;
  \item \textbf{CHARACTER VARYING:} permette di definire stringhe di caratteri
    di una lunghezza massima predefinita, di forma $VARCHAR(n)$, differisce da
    $CHAR$ nella possibilità di non dover occupare tutto lo spazio allocato.
\end{itemize}

\paragraph{Tipi temporali}%
\label{par:Tipi temporali}
\begin{itemize}
  \item \textbf{DATE:} rappresenta le date espresse come anno ($4$ cifre), mese
    ($2$ cifre) e giorno ($2$ cifre), sono disponibili diversi formati di
    rappresentazione;
  \item \textbf{TIME:} rappresenta i tempi espressi come ora ($2$ cifre), minuto
    ($2$ cifre) e secondo ($2$ cifre), la specifica ha forma $TIME[(p)]$ dove
    $p$ è l'eventuale numero di cifre frazionarie cui si è interessati ($6$
    cifre, microsecondo);
  \item \textbf{TIMESTAMP:} rappresenta una concatenazione dei tipi di dato
    $DATE$ e $TIME$, la specifica è $TIMESTAMP[(p)]$;
  \item \textbf{INTERVAL:} rappresenta una durata temporale in riferimento ad
    uno o più qualificatori tra $YEAR$, $MONTH$, $DAY$, $HOUR$, $MINUTE$ e
    $SECOND$, i valori di questo tipo sono rappresentati dalla parola chiave
    $INTERVAL$ seguita da una stringa che caratterizza la durata in termini di
    uno o più qualificatori.
\end{itemize}

\paragraph{Altri tipi definiti}%
\label{par:Altri tipi definiti}
\begin{itemize}
  \item \textbf{BOOLEAN:} i cui valori sono $TRUE$, $FALSE$, $UNKOWN$;
  \item \textbf{BLOB:} \emph{Binary Large OBject};
  \item \textbf{CLOB:} \emph{Character Large Object}.
\end{itemize}

\subsection{Creazione di relazioni}%
\label{sub:Creazione di relazioni}
In SQL la crezione di relazioni avviene in SQL tramite l'uso del comando
\lstinline[language=SQL]{CREATE TABLE}

\subsubsection{Specifica dello schema di una relazione}%
\label{ssub:Specifica dello schema di una relazione}
\begin{lstlisting}[
  language=SQL,
  escapeinside={(£}{£)},
  caption={
    Schema di creazione tabelle
  }
]
CREATE TABLE <nome relazione>
  (<specifica colonna> [,<specifica colonna>]*);
\end{lstlisting}

Dove:
\begin{itemize}
  \item \textbf{<nome relazione>:} è il nome della relazione che viene creata;
  \item \textbf{<specifica colonna>:} è una specifica di colonna, il cui formato
    è:
    \begin{lstlisting}[
      language=SQL,
      escapeinside={(£}{£)}
    ]
<nome colonna> <dominio> [DEFAULT] <valore di default>
    \end{lstlisting}
    \begin{itemize}
      \item \lstinline[language=SQL]{<nome colonna>} è il nome della colonna;
      \item \lstinline[language=SQL]{<dominio>} è il dominio della colonna e
        deve essere uno dei tipi di dato di SQL;
      \item la clausola \lstinline[language=SQL]{DEFAULT} specifica un valore di
        default per la colonna.
    \end{itemize}
\end{itemize}

\subsubsection{Vincoli di obbligatorietà, chiavi e chiavi esterne}%
\label{ssub:Vincoli di obbligatorietà}
SQL offre la possibilità di stabilire dei vincoli di obbligatorietà su colonne,
chiavi e chiavi esterne; oltre a vincoli aggiuntivi specificabili sulle tuple
della relazioni o colonne di queste tuple.

\paragraph{Obbligatorietà di colonne}%
\label{par:Obbligatorietà di colonne}
Per la specifica dell'obbligatorietà di una colonna è sufficiente aggiungere
\lstinline[language=SQL]{NOT NULL} alla specifica della colonna.

\paragraph{Chiavi}%
\label{par:Chiavi}
La specifica delle chiavi si effettua in SQL mediante le parole chiave
\lstinline[language=SQL]{UNIQUE} e \lstinline[language=SQL]{PRIMARY KEY}.

La parola chiave \lstinline[language=SQL]{UNIQUE} garantisce che non esistano
due tuple che condividano lo stesso valore.

La parola chiave \lstinline[language=SQL]{PRIMARY KEY} impone sia che i valori
non siano nulli e che non esistano due tuple che condividano lo stesso valore.

\paragraph{Chiavi esterne}%
\label{par:Chiavi esterne}
La specifica di chiavi esterne avviene mediante la clausola
\lstinline[language=SQL]{FOREIGN_KEY} ed il comando
\lstinline[language=SQL]{CREATE TABLE}, la clausola è opzionale e ripetibile.

\begin{lstlisting}[
  language=SQL,
  escapeinside={(£}{£)},
  caption={
    Sintassi per le chiavi esterne
  }
]
FOREIGN_KEY (<lista nomi colonne>)
  REFERENCES <nome relazione>
  [ON DELETE {NO ACTION | CASCADE | SET NULL | SET DEFAULT}]
  [ON UPDATE {NO ACTION | CASCADE | SET NULL | SET DEFAULT}]
\end{lstlisting}

Definisce:
\begin{itemize}
  \item una lista di una o più colonne che costituiscono una chiave esterna
    della relazione $R$;
  \item la relazione riferita $R^\prime$ di cui le suddette colonne sono chiave
    primaria;
  \item l'azione da eseguire se una tupla della relazione riferita è cancellata
    ed esistono tuple in $R$ che fanno riferimento a tale tupla, l'opzione di
    default è \lstinline[language=SQL]{NO ACTION}, le opzioni sono:
    \begin{itemize}
      \item \lstinline[language=SQL]{NO ACTION} la cancellazione di una tupla
        dalla relazione riferita è eseguita solo se non esiste alcuna tupla in
        $R$ che fa riferimento a tale tupla;
      \item \lstinline[language=SQL]{CASCADE} la cancellazione della dupla della
        relazione riferita implica la cancellazione di tutte le tuple di $R$ che
        fanno riferimento a tale tupla;
      \item \lstinline[language=SQL]{SET NULL} la cancellazione della tupla
        della relazione riferita implica che, in tutte le tuple di $R$ che fanno
        riferimento a tale tupla, il valore della chiave esterna viene posto
        uguale al valore nullo;
      \item \lstinline[language=SQL]{SET DEFAULT} la cancellazione della tupla
        della relazione riferita implica che, in tutte le tuple di $R$ che fanno
        riferimento a tale tupla, il valore della chiave esterna viene posto
        uguale al valore di default specificato per le colonne che costituiscono
        la chiave esterna.
    \end{itemize}
  \item l'azione da eseguire specificata nella clausola
    \lstinline[language=SQL]{ON UPDATE} se la chiave primaria di una tupla della
    relazione riferita viene modificata ed esistono tuple in $R$ che fanno
    riferimento a tale tupla, le azioni possibili sono le stesse precedenti,
    eccezione per l'opzione \lstinline[language=SQL]{CASCADE}, la quale ha
    l'effetto di assegnare il nuovo valore della chiave primaria di
    quest'ultima.
\end{itemize}

\subsection{Cancellazione e modifica di relazioni}%
\label{sub:Cancellazione e modifica di relazioni}
La cancellazione di una relazione è eseguita con la sintassi:
\begin{lstlisting}[
  language=SQL,
  escapeinside={(£}{£)},
  caption={
    Cancellazione di relazioni
  }
]
DROP TABLE <nome relazione> {RESTRICT | CASCADE}
\end{lstlisting}

Si rende necessario specificare aun opzione:
\begin{itemize}
  \item \lstinline[language=SQL]{RESTRICT} la relazione viene cancellata solo se
    non è riferita da altri elementi dello schema della base di dati;
  \item \lstinline[language=SQL]{CASCADE} elimina la relazione e tutti gli
    elementi dello schema che la riferiscono.
\end{itemize}

\begin{lstlisting}[
  language=SQL,
  escapeinside={(£}{£)},
  caption={
    Modifica di relazioni
  }
]
ALTER TABLE <nome relazione> <modifica>;

-- aggiunta di una colonna
ADD [COLUMN] <specifica colonna>

-- modifica di una colonna
ALTER [COLUMN] {SET DEFAULT <valore default> | DROP DEFAULT}

-- eliminazione di una colonna
DROP [COLUMN] <nome colonna> {RESTRICT | CASCADE}
\end{lstlisting}

