\section{Interrogazioni}%
\label{sec:Interrogazioni}
\subsection{SELECT}%
\label{sub:SELECT}
I comandi di interrogazione in SQL sono espressi tramite il comando
\lstinline[language=SQL]{SELECT}, ha forma:
\begin{lstlisting}[
  language=SQL,
  escapeinside={(£}{£)},
  caption={
    Forma del comando SELECT
  }
]
SELECT {DISTINCT (£$ R_{i1}.C_1,\dots,R_{in}.C_n§*$£)}
FROM (£$R_1,\dots,R_k WHERE F;$£)
\end{lstlisting}

Dove $R_i(i=1,\dots,k)$ è un nome di relazione. La clausola
\lstinline[language=SQL]{FROM} specifica pertanto le relazioni oggetto
dell'interrogazione.

L'ordine in cui le colonne sono elencate nella clausola di priezione determina
l'ordine da sinistra a destra secondo cui le colonne appaiono nella relazione
risultato.

La clausola di qualificazione può contenere i connettivi booleani
\lstinline[language=SQL]{AND,OR,NOT}.

\subsection{Operatori e funzioni}%
\label{sub:Operatori e funzioni}
SQL fornisce operatori aggiuntivi, oltre ai classici operatori relazionali di
confronto.

\subsubsection{Operatori di confronto}%
\label{ssub:Operatori di confronto}
\begin{itemize}
  \item \lstinline[language=SQL]{BETWEEN}, condizione su intervalli di valori,
    permette di trovare le tuple che contengono valori di una colonna in un
    intervallo specificato;
    \begin{lstlisting}[
      language=SQL,
      escapeinside={(£}{£)},
      caption={
        Esempio di BETWEEN
      }
    ]
    SELECT * FROM Film WHERE anno BETWEEN 1995 AND 2000;
    \end{lstlisting}
    
  \item \lstinline[language=SQL]{IN}, ricerca di valori in un insieme, permette
    di determinare le tuple che contengono uno tra i valori di un insieme
    specificato;
    \begin{lstlisting}[
      language=SQL,
      escapeinside={(£}{£)},
      caption={
        Esempio di IN
      }
    ]
    SELECT * FROM Film WHERE genere IN ('horror','fantascienza')
    \end{lstlisting}
    
  \item \lstinline[language=SQL]{LIKE}, condizioni di confronto per stringhe di
    caratteri, permette di eseguire alcune semplici operazioni di \emph{pattern
    matching} su colonne di tipo stringa, il carattera speciale $\%$ denota una
    sequenza di caratteri arbitrari di lunghezza qualsiasi, il simbolo $\_$
    indica esattamente un carattere.
    \begin{lstlisting}[
      language=SQL,
      escapeinside={(£}{£)},
      caption={
        Esempio di LIKE, match di stringhe il cui terzo carattere è d
      }
    ]
    SELECT * FROM Film WHERE titolo LIKE '__d%';
    \end{lstlisting}
\end{itemize}

\subsubsection{Espressioni e funzioni}%
\label{ssub:Espressioni e funzioni}
Le espressioni possono essere formulate applicanto funzioni alle colonne delle
tuple, usate nei predicati o comparire nella clausola di proiezione delle
interrogazioni nelle espressioni di \lstinline[language=SQL]{UPDATE}; sono
applicate su colonne \emph{virtuali}.

\paragraph{Espressioni e funzioni aritmetiche}%
\label{par:Espressioni e funzioni aritmetiche}
Oltre agli operatori aritmetici di base sono previste le funzioni:
\begin{itemize}
  \item \lstinline[language=SQL]{ABS(n)} che ha come argomento un valore
    numerico $n$ e ne calcola il valore assoluto;
  \item \lstinline[language=SQL]{MOD(n,b)} che ha come argomenti due valori
    interi $n,b$, e calcola il resto dell'intero della divisione di $n$ per $b$.
\end{itemize}

\paragraph{Espressioni e funzioni per stringhe}%
\label{par:Espressioni e funzioni per stringhe}
Il principale operatore tra stringhe è l'operatore di concatenazione $||$, sono
inoltre previste le funzioni:
\begin{itemize}
  \item \lstinline[language=SQL]{LENGTH(str)};
  \item \lstinline[language=SQL]{UPPER(str)},
    \lstinline[language=SQL]{LOWER(str)};
  \item \lstinline[language=SQL]{SUBSTR(str,m[,n])};
  \item \lstinline[language=SQL]{TRIM [str1] FROM str2}
\end{itemize}

\paragraph{Espressioni e funzioni per date e tempi}%
\label{par:Espressioni e funzioni per date e tempi}
\begin{itemize}
  \item \lstinline[language=SQL]{CURRENT_DATE};
  \item \lstinline[language=SQL]{CURRENT_TIME};
  \item \lstinline[language=SQL]{CURRENT_TIMESTAMP};
  \item \lstinline[language=SQL]{EXTRACT q FROM e}, estrae il campo
    corrispondente al qualificatore demprale $q$ dall'espressione $e$.
\end{itemize}

\subsection{Ordinamento del risultato di un'interrogazione}%
\label{sub:Ordinamento del risultato di un'interrogazione}
Il comando \lstinline[language=SQL]{SELECT} prevede la clausola addizionale
\lstinline[language=SQL]{ORDER BY} per determinare le condizioni di ordinamento.

L'ordinamento non è limitato nè ad una sola colonna nè ad un ordine crescente.
\begin{lstlisting}[
  language=SQL,
  escapeinside={(£}{£)},
  caption={
    Esempio di ORDER BY
  }
]
SELECT colloc, dataNo1, dataRest FROM Noleggio
WHERE codCli = 6635
ORDER BY dataNo1 DESC, colloc;
\end{lstlisting}

\subsection{Operazione di join}%
\label{sub:Operazione di join}
Tradizionalmente il join è espresso in SQL tramite un prodotto cartesiano a cui
sono applicati uno o più predicatu du join, tra gli operatori troviamo:
\begin{itemize}
  \item \lstinline[language=SQL]{CROSS JOIN}, corrisponde al prodotto
    cartesiano;
  \item \lstinline[language=SQL]{JOIN ON}, corrisponde al theta-join;
  \item \lstinline[language=SQL]{JOIN USING}, in cui viene richiesta
    l'uguaglianza dei valori delle colonne specificate nella clausola
    \lstinline[language=SQL]{USING};
  \item \lstinline[language=SQL]{NATURAL JOIN}, corrisponde al join naturale, in
    cui viene richiesta l'uguaglianza delle colonne che hanno lo stesso nome
    nelle due relazioni.
\end{itemize}
\begin{lstlisting}[
  language=SQL,
  escapeinside={(£}{£)},
  caption={
    Esempi di JOIN
  }
]
SELECT titolo FROM Video, Noleggio
WHERE codCli = 6635 AND dataNo1 = DATE '15-Mar-2006'
  AND Video.colloc = Noleggio.colloc;

SELECT titolo FROM Video CROSS JOIN Noleggio
WHERE codCli = 6635 AND dataNo1 = DATE '15-Mar-2006'
  AND Video.colloc = Noleggio.colloc;

SELECT titolo
  FROM Video JOIN Noleggio
    ON Video.colloc = Noleggio.colloc;
WHERE codCli = 6635 AND dataNo1 = DATE '15-Mar-2006';

SELECT titolo FROM Video JOIN Noleggio USING (colloc)
WHERE codCli = 6635 AND dataNo1 = DATE '15-Mar-2006';

SELECT titolo FROM Video NATURAL JOIN Noleggio
WHERE codCli = 6635 AND dataNo1 = DATE '15-Mar-2006';
\end{lstlisting}

\paragraph{Outer join}%
\label{par:Outer join}
In \lstinline[language=SQL]{R JOIN S} non si ha traccia delle tuple di $R$ che
non corrispondono ad alcuna tupla di $S$. Per questo motivo SQL prevede un
operatore di \lstinline[language=SQL]{OUTER JOIN} che aggiunge al risultato le
tuple di $R$ e/o $s$ che non hanno partecipato al join, completandole con
$NULL$.

L'operatore di join originario, per contrasto, viene anche detto
\lstinline[language=SQL]{INNER JOIN}.

Esistono diverse varianti dell'outer join, che vengono specificate permettendo
il corrispondente qualificatore all'operatore
\lstinline[language=SQL]{OUTER JOIN}.
Consideriamo \lstinline[language=SQL]{R OUTER JOIN S}:
\begin{itemize}
  \item \lstinline[language=SQL]{FULL}, sia le tuple di $R$ sia quelle di $S$
    che non partecipano al join vengono completate ed inserite nel risultato;
  \item \lstinline[language=SQL]{LEFT}, le tuple di $R$ che non partecipano al
    join vengono completate ed inserite nel risultato;
  \item \lstinline[language=SQL]{RIGHT}, le tuple di $S$ che non partecipano al
    join vengono completate ed inserite nel risultato.
\end{itemize}

\begin{lstlisting}[
  language=SQL,
  escapeinside={(£}{£)},
  caption={
    Esempio di OTER JOIN
  }
]
SELECT colloc, titolo, codCli
FROM Film NATURAL JOIN Video NATURAL LEFT OUTER JOIN Noleggio
WHERE regista = 'tim burton' AND genere = 'fantastico';
\end{lstlisting}

